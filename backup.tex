\section{TODO}

\subsection{Max cut}
De-Bruijn graphs are claw-free, and we can find maximum independent sets (MIS) in claw-free graphs in polynomial time. If $H$ is the line graph of $G$, $\texttt{MIS}(H)$ corresponds to a directed cut in $G$. Since we've established that cuts in the de-Bruijn graph are symmetric, then a directed cut with $x$ edges corresponds to a undirected cut with $2x$ edges. Therefore, if $MIS(G_{w+1})=x$, then $\texttt{MC}(G_w)=2x$.



\begin{itemize}
    \item Can we efficiently generate a max cut for a de-bruijn graph? 
    \item Can we do it implicitly? 
    \item Can we determine the max-cut size in O(poly(w)) memory?
\end{itemize}

\subsection{Extending to any $r$ for forward schemes}
This concept extends to $(w, r)$-forward schemes, where we aim to maximize the number of "consistent" edges $(u, v)$ where $f(u)-1 = f(v)$. The question is how do we maximize consistent edges while also prohibiting edges $(u, v)$ where $f(u)-1 > f(v)$? Alternatively, we can extend the idea of "consistent" edges to local schemes, dropping the forward requirement. In that case, using the number of consistent/charged edges to determine density would lead to an upper bound on density, as a "charged" edge doesn't always lead to a newly selected position. For example, 3->1->1 has 2 charged edges, but only 2 sampled positions. 

There is also the question of when the optimal forward scheme as good as the optimal local scheme?